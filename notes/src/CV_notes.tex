% \documentclass[a4,10pt,oneside,openany]{ctexart}
\usepackage[utf8x]{inputenc}
%\usepackage[UTF8]{ctex}
%\usepackage[english]{babel}
\usepackage{lmodern}
\usepackage{mathpazo}
\usepackage{mathrsfs}
\setlength{\parskip}{0.2\baselineskip}
%\setlength{\parskip}{0.2em}
\usepackage{url}
\usepackage{amssymb,amsmath,amsthm,amsfonts}
\usepackage{mathtools}
\usepackage{geometry}
\usepackage{float}
\usepackage{tikz}
\tikzset{elegant/.style={smooth,thick,samples=50,cyan}}
\tikzset{eaxis/.style={->,>=stealth}}


\usepackage{extarrows}
\usepackage{subfigure}
\usepackage{graphics}
\usepackage{wrapfig}
\usepackage{graphicx}
\usepackage{enumerate}
\usepackage{subfigure}
\usepackage{CJKulem}%中文删除线
\usepackage[normalem]{ulem} % \sout{想加删除线的中文}
\usepackage{picinpar}%图片绕排,是真滴烦
\usepackage{float}

\usepackage{fontspec}
\setmainfont{Latin Modern Roman}

%页眉页脚
\usepackage{authblk}
\usepackage{fancyhdr}
\usepackage{lastpage}

%以下是书写伪代码用
\usepackage{algorithm, algorithmicx}  
\usepackage{algpseudocode}  
\renewcommand{\algorithmicrequire}{\textbf{Input:}}  % Use Input in the format of Algorithm  
\renewcommand{\algorithmicensure}{\textbf{Output:}} % Use Output in the format of Algorithm 

\geometry{a4paper, left=3cm, right=3cm, bottom=3cm, top=3cm}%此行定义了纸张大小和边距,不需要可删除

%以下是Inkspace
\usepackage{import}
\usepackage{xifthen}
\usepackage{pdfpages}
\usepackage{transparent}

\usepackage{verbatim}%comment 环境
%\usepackage[backref]{hyperref} %参考文献高亮跳转\\
\usepackage[colorlinks, linkcolor=black, anchorcolor=blue, citecolor=red]{hyperref}

\usepackage{chngcntr}
\counterwithin{figure}{section}

\newcommand{\incfig}[1]{%
\def\svgwidth{\columnwidth}
\import{./figures/}{#1.pdf_tex}
}

\newtheorem{deff}{定义}[section]
\newtheorem{thm}{定理}[section]
\newtheorem{clm}[thm]{命题}
\newtheorem{lemma}[thm]{引理}
\newtheorem{prf}{证明}[section]

\renewcommand{\arraystretch}{1.5} % Adjust the value as needed

\renewcommand{\proofname}{\indent Pr}
\renewcommand{\qedsymbol}{$\blacksquare$}    % 证毕符号改成黑色的正方形

\newcommand{\argmin}[1]{\underset{#1}{\arg \min}\ }
\newcommand{\ceil}[1]{\left\lceil #1 \right \rceil }
\newcommand{\norm}[1]{\left \Vert #1 \right \Vert}
\newcommand{\tform}[1]{\left \Vert #1 \right \Vert_2}
\newcommand{\tnorm}[1]{\left \Vert #1 \right \Vert_2}
\newcommand{\onorm}[1]{\left \Vert #1 \right \Vert_1}
\newcommand{\abs}[1]{\left|#1 \right|}
\newcommand{\var}[1]{\text{Var}\left[ #1\right]}
\newcommand{\xk}[1]{\left( #1\right)} 
\newcommand{\zk}[1]{\left[ #1\right]} 
\newcommand{\dk}[1]{\left\{ #1\right\}} 
\newcommand{\bd}[1]{\bold{#1}}

\newcommand{\R}{\mathbb{R}}
\newcommand{\N}{\mathbb{N}}
\newcommand{\Z}{\mathbb{Z}}
\newcommand{\C}{\mathbb{C}}

%量子力学符号------
\newcommand{\xde}{\text{Schrödinger}}
\newcommand{\avg}[1]{\left \langle #1 \right \rangle}
\newcommand{\lvec}[1]{\left \langle #1 \right |}
\newcommand{\rvec}[1]{\left | #1 \right \rangle}


\newtheorem{lproof}{证明}[section]
\newtheorem{tuilun}{推论}
\newtheorem{eg}{例}[section]
\newtheorem{solve}{解}[section]
\newcommand\ii{\textup{i}}
\newcommand\dd{\mathrm{d}}


%自定义数学符号
\newcommand{\diag}{\textup{diag}}
\newcommand{\Frobenius}[1]{\left\Vert #1 \right\Vert}
\newcommand{\fform}[1]{\left\Vert #1 \right\Vert_F}
\newcommand{\parr}[2]{\frac{\partial #1}{\partial #2}}%一阶偏微分
\newcommand{\parrr}[2]{\frac{\partial^2 #1}{\partial #2^2}}%二阶偏微分
\newcommand{\lap}[1]{\parrr{#1}{x} + \parrr{#1}{y} = 0}%二元拉普拉斯方程
\newcommand{\ddd}[2]{\frac{\textup{d} #1}{\textup{d} #2}}%微商
\newcommand{\dddd}[2]{\frac{\textup{d}^2 #1}{\textup{d} #2^2}}%微商

%二重以上环路积分,强迫症了属于是
\def\ooint{{\bigcirc}\kern-11.5pt{\int}\kern-6.5pt{\int}}
\def\oooint{{\bigcirc}\kern-12.3pt{\int}\kern-7pt{\int}\kern-7pt{\int}}


\newcommand{\tu}{\textup}
\newcommand{\bm}{\boldsymbol}
\newcommand{\ol}[1]{$\overline{#1}$}
\newcommand{\re}[1]{\textup{Re}(#1)}
\newcommand{\im}[1]{\textup{Im}(#1)}
\newcommand{\fa}{\forall}
\newcommand{\ex}{\exists}
\newcommand{\st}{\textup{  s.t. }}
\newcommand{\ve}{\varepsilon}
\newcommand{\disp}{\displaystyle}
\newcommand{\chj}{\textup{Cauchy}积分公式}
\newcommand{\res}[1]{\textup{Res}\left(#1\right)}
\newcommand{\mysum}[1][n]{\sum_{i = 1}^{#1}}%求和
\newcommand{\series}[1]{\sum_{n = 0}^{\infty} #1_{n}}%级数
\newcommand{\seriesa}[1]{\sum_{n = 0}^{\infty} \left| #1_{n}\right|}%绝对级数
\newcommand{\fseries}[1]{\sum_{k = 1}^{\infty} #1_k (z)}

\newcommand*{\num}{pi}

%书写横线
\newcommand{\horrule}[1]{\rule[0.5ex]{\linewidth}{#1}} 	% Horizontal rule

\RequirePackage{algorithm}

\makeatletter
\newenvironment{algo}
  {% \begin{breakablealgorithm}
    \begin{center}
      \refstepcounter{algorithm}% New algorithm
      \hrule height.8pt depth0pt \kern2pt% \@fs@pre for \@fs@ruled
      \parskip 0pt
      \renewcommand{\caption}[2][\relax]{% Make a new \caption
        {\raggedright\textbf{\fname@algorithm~\thealgorithm} ##2\par}%
        \ifx\relax##1\relax % #1 is \relax
          \addcontentsline{loa}{algorithm}{\protect\numberline{\thealgorithm}##2}%
        \else % #1 is not \relax
          \addcontentsline{loa}{algorithm}{\protect\numberline{\thealgorithm}##1}%
        \fi
        \kern2pt\hrule\kern2pt
     }
  }
  {% \end{breakablealgorithm}
     \kern2pt\hrule\relax% \@fs@post for \@fs@ruled
   \end{center}
  }
\makeatother

\renewcommand{\arraystretch}{0.85}

\documentclass[a4,10pt,oneside,openany]{ctexart}
\usepackage[utf8x]{inputenc}
%\usepackage[UTF8]{ctex}
%\usepackage[english]{babel}
\usepackage{lmodern}
\usepackage{mathpazo}
\usepackage{mathrsfs}
\setlength{\parskip}{0.2\baselineskip}
%\setlength{\parskip}{0.2em}
\usepackage{url}
\usepackage{amssymb,amsmath,amsthm,amsfonts}
\usepackage{mathtools}
\usepackage{geometry}
\usepackage{float}
\usepackage{tikz}
\tikzset{elegant/.style={smooth,thick,samples=50,cyan}}
\tikzset{eaxis/.style={->,>=stealth}}


\usepackage{extarrows}
\usepackage{subfigure}
\usepackage{graphics}
\usepackage{wrapfig}
\usepackage{graphicx}
\usepackage{enumerate}
\usepackage{subfigure}
\usepackage{CJKulem}%中文删除线
\usepackage[normalem]{ulem} % \sout{想加删除线的中文}
\usepackage{picinpar}%图片绕排,是真滴烦
\usepackage{float}

\usepackage{fontspec}
\setmainfont{Latin Modern Roman}

%页眉页脚
\usepackage{authblk}
\usepackage{fancyhdr}
\usepackage{lastpage}

%以下是书写伪代码用
\usepackage{algorithm, algorithmicx}  
\usepackage{algpseudocode}  
\renewcommand{\algorithmicrequire}{\textbf{Input:}}  % Use Input in the format of Algorithm  
\renewcommand{\algorithmicensure}{\textbf{Output:}} % Use Output in the format of Algorithm 

\geometry{a4paper, left=3cm, right=3cm, bottom=3cm, top=3cm}%此行定义了纸张大小和边距,不需要可删除

%以下是Inkspace
\usepackage{import}
\usepackage{xifthen}
\usepackage{pdfpages}
\usepackage{transparent}

\usepackage{verbatim}%comment 环境
%\usepackage[backref]{hyperref} %参考文献高亮跳转\\
\usepackage[colorlinks, linkcolor=black, anchorcolor=blue, citecolor=red]{hyperref}

\usepackage{chngcntr}
\counterwithin{figure}{section}

\newcommand{\incfig}[1]{%
\def\svgwidth{\columnwidth}
\import{./figures/}{#1.pdf_tex}
}

\newtheorem{deff}{定义}[section]
\newtheorem{thm}{定理}[section]
\newtheorem{clm}[thm]{命题}
\newtheorem{lemma}[thm]{引理}
\newtheorem{prf}{证明}[section]

\renewcommand{\arraystretch}{1.5} % Adjust the value as needed

\renewcommand{\proofname}{\indent Pr}
\renewcommand{\qedsymbol}{$\blacksquare$}    % 证毕符号改成黑色的正方形

\newcommand{\argmin}[1]{\underset{#1}{\arg \min}\ }
\newcommand{\ceil}[1]{\left\lceil #1 \right \rceil }
\newcommand{\norm}[1]{\left \Vert #1 \right \Vert}
\newcommand{\tform}[1]{\left \Vert #1 \right \Vert_2}
\newcommand{\tnorm}[1]{\left \Vert #1 \right \Vert_2}
\newcommand{\onorm}[1]{\left \Vert #1 \right \Vert_1}
\newcommand{\abs}[1]{\left|#1 \right|}
\newcommand{\var}[1]{\text{Var}\left[ #1\right]}
\newcommand{\xk}[1]{\left( #1\right)} 
\newcommand{\zk}[1]{\left[ #1\right]} 
\newcommand{\dk}[1]{\left\{ #1\right\}} 
\newcommand{\bd}[1]{\bold{#1}}

\newcommand{\R}{\mathbb{R}}
\newcommand{\N}{\mathbb{N}}
\newcommand{\Z}{\mathbb{Z}}
\newcommand{\C}{\mathbb{C}}

%量子力学符号------
\newcommand{\xde}{\text{Schrödinger}}
\newcommand{\avg}[1]{\left \langle #1 \right \rangle}
\newcommand{\lvec}[1]{\left \langle #1 \right |}
\newcommand{\rvec}[1]{\left | #1 \right \rangle}


\newtheorem{lproof}{证明}[section]
\newtheorem{tuilun}{推论}
\newtheorem{eg}{例}[section]
\newtheorem{solve}{解}[section]
\newcommand\ii{\textup{i}}
\newcommand\dd{\mathrm{d}}


%自定义数学符号
\newcommand{\diag}{\textup{diag}}
\newcommand{\Frobenius}[1]{\left\Vert #1 \right\Vert}
\newcommand{\fform}[1]{\left\Vert #1 \right\Vert_F}
\newcommand{\parr}[2]{\frac{\partial #1}{\partial #2}}%一阶偏微分
\newcommand{\parrr}[2]{\frac{\partial^2 #1}{\partial #2^2}}%二阶偏微分
\newcommand{\lap}[1]{\parrr{#1}{x} + \parrr{#1}{y} = 0}%二元拉普拉斯方程
\newcommand{\ddd}[2]{\frac{\textup{d} #1}{\textup{d} #2}}%微商
\newcommand{\dddd}[2]{\frac{\textup{d}^2 #1}{\textup{d} #2^2}}%微商

%二重以上环路积分,强迫症了属于是
\def\ooint{{\bigcirc}\kern-11.5pt{\int}\kern-6.5pt{\int}}
\def\oooint{{\bigcirc}\kern-12.3pt{\int}\kern-7pt{\int}\kern-7pt{\int}}


\newcommand{\tu}{\textup}
\newcommand{\bm}{\boldsymbol}
\newcommand{\ol}[1]{$\overline{#1}$}
\newcommand{\re}[1]{\textup{Re}(#1)}
\newcommand{\im}[1]{\textup{Im}(#1)}
\newcommand{\fa}{\forall}
\newcommand{\ex}{\exists}
\newcommand{\st}{\textup{  s.t. }}
\newcommand{\ve}{\varepsilon}
\newcommand{\disp}{\displaystyle}
\newcommand{\chj}{\textup{Cauchy}积分公式}
\newcommand{\res}[1]{\textup{Res}\left(#1\right)}
\newcommand{\mysum}[1][n]{\sum_{i = 1}^{#1}}%求和
\newcommand{\series}[1]{\sum_{n = 0}^{\infty} #1_{n}}%级数
\newcommand{\seriesa}[1]{\sum_{n = 0}^{\infty} \left| #1_{n}\right|}%绝对级数
\newcommand{\fseries}[1]{\sum_{k = 1}^{\infty} #1_k (z)}

\newcommand*{\num}{pi}

%书写横线
\newcommand{\horrule}[1]{\rule[0.5ex]{\linewidth}{#1}} 	% Horizontal rule

\RequirePackage{algorithm}

\makeatletter
\newenvironment{algo}
  {% \begin{breakablealgorithm}
    \begin{center}
      \refstepcounter{algorithm}% New algorithm
      \hrule height.8pt depth0pt \kern2pt% \@fs@pre for \@fs@ruled
      \parskip 0pt
      \renewcommand{\caption}[2][\relax]{% Make a new \caption
        {\raggedright\textbf{\fname@algorithm~\thealgorithm} ##2\par}%
        \ifx\relax##1\relax % #1 is \relax
          \addcontentsline{loa}{algorithm}{\protect\numberline{\thealgorithm}##2}%
        \else % #1 is not \relax
          \addcontentsline{loa}{algorithm}{\protect\numberline{\thealgorithm}##1}%
        \fi
        \kern2pt\hrule\kern2pt
     }
  }
  {% \end{breakablealgorithm}
     \kern2pt\hrule\relax% \@fs@post for \@fs@ruled
   \end{center}
  }
\makeatother

\renewcommand{\arraystretch}{0.85}

%%%%%%%%%%%% Above is the preamble %%%%%%%%%%%


% 文档标题
\title{
{\normalfont\normalsize\textsc{
Peking University\\
Introduction to Computer Vision, Spring 2025 \\[25pt]}}
\horrule{0.5pt}\\
\sffamily{Introduction to Computer Vision\\Course Notes}\\
\horrule{1.8pt}\\[20pt]
}

% 作者和联系方式
\author[1]{Prof. He Wang\thanks{\href{https://hughw19.github.io/}{Prof. He Wang}}}
\author[2]{林晓疏\thanks{wangyuanqing@pku.edu.cn}}
\author[2]{Yutong Liang\thanks{\href{https://lyt0112.com/}{Yutong Liang's Website}}}
\author[2]{Jiacong Fang\thanks{jiacong\_fang@stu.pku.edu.cn}}
\affil[1]{主讲教师}
\affil[2]{笔记整理}

% 文档日期
\date{\today}

\pagestyle{fancy}
\fancyhf{}
\fancyhead[L]{\leftmark}  % 在页眉左侧显示章节名
\fancyfoot[C]{\thepage}  % 在页脚中间显示页码


\begin{document}
	\maketitle
	\include{preface}
	\clearpage
	\tableofcontents
	\section{Image as Functions}
\label{sec:image_as_functions}

\subsection{A Brief History}

\subsection{Image as Functions}
图片定义成一个函数 $f: \R^2 \to \R^M$ ,其中 $M$ 是颜色通道的数量,通常是 1 (灰度图) 或 3 (RGB 图). 
\begin{itemize}
    \item \textbf{Pixel value}: intensity in $[0, 255]$ and color with 3 channels $[R, G, B]$.
    \item \textbf{Resolution}: $H \times W$. 
    \item An image contains discrete number of pixels $\to$ A tensor with $[H, W, C]$.
\end{itemize}

\subsection{Image Gradient}
Image $f(x, y)$ $\implies$ Gradient $\nabla f(x, y) = \left[ \frac{\partial f}{\partial x}, \frac{\partial f}{\partial y} \right]$.

在实际中是一个有限差分的近似:
\begin{equation}
    \frac{\partial f}{\partial x} \bigg|_{(x, y)=(x_0, y_0)} \approx \frac{f(x_0 + 1, y_0) - f(x_0 - 1, y_0)}{2}
\end{equation}

对于图片我们将梯度可视化可以得到 Gradient Maginitude, 间接反映图片的纹理和边缘信息.
\[
    \text{Gradient Magnitude} = \norm{\nabla f} = \sqrt{\left( \frac{\partial f}{\partial x} \right)^2 + \left( \frac{\partial f}{\partial y} \right)^2}
\]
% TODO: Add a figure to illustrate the gradient magnitude.

\subsection{(Signal) Convolution Filter}
\subsubsection{Quick Facts of Convolution}
下图展示了离散信号和连续信号的卷积操作:
\[
\begin{array}{ccc}
    \hline
    & \text{Discrete Signal} & \text{Continuous Signal} \\
    \hline
    \text{Convolution} & (f * g)[n] = \sum_{i = -\infty}^{\infty} f[i] g[n - i] & (f * g)(t) = \int_{-\infty}^{\infty} f(\tau) g(t - \tau) d\tau \\
    \text{Fourier Transform} & \mathcal{F}[n] = \sum_{m = 0}^{M-1} f[m] \exp\left( -\frac{2\pi i}{M} m n \right) & \mathcal{F}(f) = \int_{-\infty}^{\infty} f(t) \exp(-2\pi i \omega t) dt \\ 
    \hline
\end{array}
\]

对于卷积操作最重要的性质是 \textbf{Convolution Theorem}, 它告诉我们在时域上的卷积等价于频域上的乘积\marginpar{\kaishu Fourier变换将时域信号转化为频域信号}, 即:
\[
    \mathcal{F}(f * g) = \mathcal{F}(f) \cdot \mathcal{F}(g) \implies f * g = \mathcal{F}^{-1}(\mathcal{F}(f) \cdot \mathcal{F}(g))
\]

这个定理使得我们可以使用傅立叶变换与逆变换来便捷的计算卷积操作. 此外, 对于卷积的微分有如下性质
\[
    \frac{\dd }{\dd x} (f * g) = f * \frac{\dd }{\dd x} g
\]

按照傅立叶变换后的滤波函数的集中情况, 可以将滤波器分为:
\begin{itemize}
    \item 低通滤波器(Low-pass Filter) 滤去高频信号, 保留低频信号. 
    \item 高通滤波器(High-pass Filter) 滤去低频信号, 保留高频信号.
    \item 带通滤波器(Band-pass Filter) 只保留某个频率段的信号.
    \item 带阻滤波器(Band-stop Filter) 只滤去某个频率段的信号.
\end{itemize}
\begin{figure}[htbp]
    \centering
    \includegraphics[scale=0.75]{figures/Fourier.png}
    \caption{Rectangular Function and its Fourier Transform(A low-pass filter)}
\end{figure}


\subsubsection{1D Discrete-Space Filter} 
在一维上信号 $f[n]$ 经过系统 $\mathcal{G}$ 后得到输出 $h[n]$, 可以表示为:
\[
    f[n] \to \fbox{\text{System } $\mathcal{G}$} \to h[n], \quad \text{Output: } h[h] = \mathcal{G}(f)[n]
\]

例如 Moving Average Filter, 在局部区域内取平均值 (假设前后的噪声是独立的, 通过平均值可以使得噪声相互抵消.)
\[
    h[n] = \frac{1}{k} \sum_{i = n}^{n + k} f[i]
\]

能够抑制高斯噪声, 但是会使得信号的边缘变得平滑(直接跳迁变为渐变). 更一般的可以定义为卷积(Convolution)操作:
\[
    h[n] = (f * g)[n] = \sum_{i = -\infty}^{\infty} f[i] g[n - i]
\] 

其中 $g[n]$ 是卷积核, 也叫做滤波器(Filter). \marginpar{\kaishu Move Average Filter 是一种低通滤波器, 可以滤去高频噪声, 而保留低频原始信号.}

如何解决信号的边缘平滑问题? $\implies$ 设计更好的滤波器. $\implies$ 在深度学习中, 通过 Learning 得到更好的滤波器/卷积核.

\subsubsection{2D Discrete-Space Filter}
对于二维信号 $f(x, y)$, 这里简单列出卷积操作的定义:
\begin{equation*}
    f[n, m] \to \fbox{\text{System } $\mathcal{G}$} \to h[n, m], \quad \text{Output: } h[h, m] = \mathcal{G}(f)[n, m]
\end{equation*}
\begin{equation*}
    h[n, m] = (f * g)[n, m] = \sum_{k, l} f[k, l] g[n - k, m - l]
\end{equation*}

使用一个全1的卷积核(即二维下的 Moving Average Filter)可以得到一个平滑的图像, 但是会使得图像的边缘变得模糊. \marginpar{\kaishu 边缘附近像素会发生跳变, 是高频信号}

这里我们可以使用 Binarization via Thresholding (阈值二值化, Non-linear Filter) 来处理上述边缘模糊问题. 即设置阈值 $\tau$, 将大于 $\tau$ 的像素设为 1, 小于 $\tau$ 的像素设为 0.
\[
    h[n, m] = \begin{cases}
        1, & \text{if } f[n, m] > \tau \\
        0, & \text{otherwise}
    \end{cases}
\]
\marginpar{\kaishu 图为燕园的猫雪风}
\begin{figure}[htbp]
    \centering
    \includegraphics[scale=0.35]{figures/cat_gaussian.png}
    \caption{Visualizing the effect of Gaussian Filter and Binarization}
\end{figure}


	\section{Edge Detection}

\subsection{What is an Edge?}
\marginpar{\kaishu Step 1: 问题定义, problem formulation.}
“边缘”是图像中的一个区域,在这个区域中,沿着图像的一个方向,
像素强度值 (或者说对比度) 发生了“显著(Significant)”的变化,而在其正交方向上,
像素强度值 (或对比度) 几乎没有变化.

注意: Gradient magnitude 大的地方不一定是边缘, 可能是多个方向差分都有显著变化, 即边缘交叉处或者可能是噪声.

\subsection{Criteria for Optimal Edge Detection}
\marginpar{\kaishu Step 2: 评价标准, evaluation metrics.}
我们使用 Accuract, Precision, Recall 来评价边缘检测的好坏.
\[
\begin{array}{c|cc}
    \hline
     & \text{Postive} & \text{Negative} \\
    \hline
    \text{True} & \text{True Positive (TP)} & \text{False Negative (FN)} \\
    \text{False} & \text{False Positive (FP)} & \text{True Negative (TN)} \\
    \hline
\end{array}
\]

具体定义如下:
\begin{equation}
\text{Accuracy}=\frac{\text{TP}+\text{TN}}{\text{TP}+\text{FP}+\text{TN}+\text{FN}} 
\end{equation}

\begin{equation}
\text{Precision}=\frac{\text{TP}}{\text{TP}+\text{FP}} 
\end{equation}

\begin{equation}
\text{Recall}=\frac{\text{TP}}{\text{TP}+\text{FN}}
\end{equation}

Precision 和 Recall 都代表着你检测出的真正边缘所占比例,但是 Precision 的分母
是你检测出的边缘,Recall 的分母是真正的边缘.好的边缘检测算法应满足:
\begin{itemize}
    \item High precision: 确保所有检测到的边都是正确的 (via minimizing FP).
    \item High recall: 确保所有边都被检测到 (via minimizing FN).
    \item Good localization: 边缘检测的边缘应该尽可能的靠近真实边缘.
    \item Single response constraint: 最小化响应的数量.
\end{itemize}

在后续物体检测任务中会定义Precision-Recall Curve 和 Average Precision, 以综合考虑 Precision 和 Recall.

\subsection{Smoothing by Gaussian Filter}
由于上述提及的 Gradient magnitude 在每个像素点的值都非零, 且梯度对于噪声敏感, 难以从中直接将
边缘提取出来, 所以我们需要对图像进行平滑处理. 

这里我们使用 Gaussian Filter, 它是一个低通滤波器, 且Gaussian kernel的傅立叶变换是一个还是一个 Gaussian kernel, 即
\[
    g = \frac{1}{\sqrt{2\pi}\sigma} \exp\left( -\frac{x^2}{2\sigma^2} \right), \quad \mathcal{F}(g) = \exp \left( -\frac{\sigma^2\omega^2}{2} \right)
\]
\begin{itemize}
    \item 当 $\sigma$ 越大时, $\mathcal{F}(g)$ 越集中于低频, 当 $\sigma \to \infty$ 时, 会滤过所有高频只留下常数信号.
    \item 当 $\sigma$ 越小时, $\mathcal{F}(g)$ 越分散, 会保留更多高频信号.
\end{itemize}
在实际运算中会用到一个小技巧来减少计算量, Derivative Theorem of Convolution:
\[
\frac{\dd }{\dd x} (f * g) = f * \frac{\dd }{\dd x} g
\]
那么可以将运算从两步(先求导, 再卷积)变为一步(直接卷积).


\subsection{Non-Maximal Suppression (NMS)}

非最大值抑制,顾名思义,就是抑制非最大值,这里的最大值指的是梯度的局部最大值.

在计算出了所有点的梯度之后,会有很多像素的梯度大于设定的阈值,而我们希望最后得出的边缘像素真的看起来
像一条线而不是一块区域,所以 NMS 的目的是为了抑制那些不是边缘的像素,只保留那些是边缘的像素.

\begin{figure}[htbp]
    \centering
	\includegraphics[scale=0.2]{figures/NMS.png}
	\caption{NMS示意图}
\end{figure}

对于一个边缘像素的候选点,我们认为它是边缘当:它比它梯度方向的两个点 $q+\nabla q$ 和 $q-\nabla q$ 的梯度值大,
也就是这个点的梯度大小是局部最大值的时候.

\begin{figure}[htbp]
    \centering
	\includegraphics[scale=0.4]{figures/bilinear.png}
	\caption{双线性插值}
\end{figure}

计算这个点梯度方向的点的梯度值可以使用双线性插值法,就是把这个点周围的四个点的梯度按照横纵距离反比加权.

当然,NMS 是一个思想而不是针对边缘检测的算法,比如对于 keypoint detection,object detection (like YOLO) 都可以使用 NMS,
实现的思路都很类似,使用一个打分函数看这个备选点 (bounding box) 是不是比跟它相邻 (冲突) 的点 (bounding box) 好,如果是就保留,否则就抑制.

\subsection{A Simplified Version of NMS}

\begin{figure}[htbp]
    \centering
	\includegraphics[scale=0.55]{figures/simple_NMS.png}
	\caption{简化版本的双线性插值}
\end{figure}

一个 NMS 的简化版本是把双线性插值省去,直接让这个像素的梯度大于它梯度方向的那两个相邻像素的梯度.

\subsection{Edge Linking: Hysteresis Thresholding}
遍历每个像素点, 使用高阈值 (maxVal) 开始边缘曲线,使用低阈值 (minVal) 继续它们.

\begin{itemize}
    \item Pixels with gradient magnitudes > maxVal should be reserved.
    \item Pixels with gradient magnitudes < minVal should be removed.
\end{itemize}

How to decide maxVal and minVal? Examples:

\begin{itemize}
    \item maxVal = 0.3 $\times$ average magnitude of the pixels that pass NMS
    \item minVal = 0.1 $\times$ average magnitude of the pixels that pass NMS
\end{itemize}
\begin{figure}[htbp]
    \centering
    \includegraphics[scale=0.4]{figures/canny.png}
    \caption{Canny Edge Detection with Different $\sigma$}
\end{figure}
上述过程就是经典的 Canny Edge Detection 算法, 算法的核心是使用了 Gaussian Filter, 超参数为 $\sigma$, $\sigma$ 越大时, 越会关注更显著的边缘.

	\section{Line Fitting}
\subsection{Least Square Method}
\begin{itemize}
    \item Data points: $(x_i, y_i)$, $i=1,2,\ldots,n$
    \item Line equation: $y_i - mx_i - b = 0$
    \item Objective: minimize the sum of squared errors ($\mathcal{L}_2$ norm): 
    \[
    \text{Energy} = \sum_{i=1}^{n} (y_i - mx_i - b)^2
    \]
\end{itemize}
对于不同的范数选择, energy landscape 会有所不同, 使用 $\mathcal{L}_2$ norm 我们可以得到一个解析解:
\begin{align*}
    E = \norm{ \mathbf{Y} - \mathbf{X} \mathbf{B} }^2 &= T^T Y - 2 (XB)^T Y + (XB)^T (XB) \\
    \frac{\partial E}{\partial B} &= - 2 X^T Y + 2 X^T X B = 0  \\
    \implies B^* &= (X^T X)^{-1} X^T Y \quad \text{where } B := [m, b]^T
\end{align*}
注: 关于矩阵求导可以参考 \href{https://www.math.uwaterloo.ca/~hwolkowi/matrixcookbook.pdf}{Matrix Cookbook} 或 \href{https://ccrma.stanford.edu/~dattorro/matrixcalc.pdf}{Matrix Calculus}.
\subsubsection{SVD Decomposition}
原始方法对于垂直线失效。$\implies$ 因此改用 $ax + by + d = 0$,此时优化问题变为:\begin{align*}
    \text{minimize} &\quad \norm{A h} = 0 \\ 
    \text{subject to } &\quad \norm{h} = 1
\end{align*}
其中 
\[
    \renewcommand{\arraystretch}{0.85}
    A = \begin{bmatrix}
        x_1 & y_1 & 1 \\
        x_2 & y_2 & 1\\
        & \cdots &  \\
        x_n & y_n & 1
    \end{bmatrix}, \quad h = \begin{bmatrix}
        a \\ b \\ d
    \end{bmatrix}
\]
我们可以使用 SVD 分解来求解 $h$. \marginpar{\kaishu 关于奇异值分解的详细内容可以参考附录.}
\[
    A_{n\times 3} = U_{n\times n} \Sigma_{n\times 3} V_{3\times 3}^T 
\]
其中 $U_{n\times n}$ 和 $V_{3\times 3}$ 是正交矩阵, 不妨设
\[
    \Sigma = \begin{bmatrix}
        \diag\{\lambda_1, \lambda_2, \lambda_3\} \\
        0
    \end{bmatrix}, \quad |\lambda_1| > |\lambda_2| > |\lambda_3|
\]
设 $V = [c_1, c_2, c_3]$, 其中 $\{c_i\}$ 构成了 $R^3$ 的正交基, 那么存在 $\alpha_1, \alpha_2, \alpha_3$ 使得
$ h = \alpha_1 c_1 + \alpha_2 c_2 + \alpha_3 c_3$ 且 $\alpha_1^2 + \alpha_2^2 + \alpha_3^2 = 1$.
进而:
\begin{align*}
    Ah = U \Sigma V^T h = U \Sigma \begin{bmatrix}
        \alpha_1 \\ \alpha_2 \\ \alpha_3
    \end{bmatrix} = \begin{bmatrix}
        \diag\{\lambda_1 \alpha_1, \lambda_2 \alpha_2, \lambda_3 \alpha_3\} \\
        0
    \end{bmatrix} \implies \norm{Ah}^2 = \sum_{i=1}^{3} \lambda_i^2 \alpha_i^2 \ge \lambda_3^2
\end{align*} 
注意到在 $h = c_3$ 时取等号, 因此最优解为 $h = c_3$.


\subsubsection{Robustness}
Least square method is robust to small noises but sensitive to outliers.

$\mathcal{L}_1$ norm 对于outlier的鲁棒性会更强(梯度与残差的大小不相关), 但优化问题更困难.
\begin{figure}[htbp]
    \centering
    \includegraphics[width=0.8\textwidth]{figures/not_roboust_outliner.png}
    \caption{Least Square Method is not robust to outliers}
\end{figure}


\subsection{RANSAC}

RANSAC:RANdom SAmple Consensus
\subsubsection{Ideas and Algorithm}
\textbf{Idea:} we need to find a line that has the largest supporters (or inliers)
\marginpar{\kaishu 在作业中会实现并行算法.}
\begin{algo}
    \centering 
    \caption{\textbf{RANSAC Loop(Sequential Version)}}
    \begin{algorithmic}[1]
        \Require 假设这个直线 (平面) 需要两个 ($n$个) 点来确定. 阈值 $\delta$, $\gamma$.
        \Ensure 最优的直线 (平面) 参数
        \State 随机选择 $k$ 组能确定这个直线的点,也就是在所有点里面选出一个 $k\times 2$ 的矩阵.
        \State 对每一组点计算出一条直线
        \State 对每一组点的直线计算出所有点到这条直线的距离,如果小于阈值 $\delta$,则认为这个点是这条直线的 inlier
        \State 找到最大的 inlier 数量的直线,如果大于阈值 $\gamma$,则认为这条直线是最优的
        \State 对这个最优的直线,用这个直线所有的 inlier 重新使用最小二乘法计算出最优的直线参数 
    \end{algorithmic}
\end{algo}
{\kaishu 注: 可能会出现多组直线的 inliner 数量相同的情况, 此时使用所有inlier进行最小二乘法拟合是必要的.}

\subsubsection{How Many Samples?}

假设我们有所有 inliner 占比为 $w$ 的先验知识,同时希望有不低于 $p$ 的概率能够找到一个最优的直线,那么我们需要多少次迭代呢?

\begin{equation}
\mathbf{\Pr}\text{[$n$个点组成的 sample 全部是inliner]} = w^n
\end{equation}

如果一组点中有一个点是 outliner,那么我们称这组点 fail.

\begin{equation}
\mathbf{\Pr}\text{[k组点全部fail]} = {(1-w^n)}^k
\end{equation}

我们希望 k 组点全部 fail 的概率小于 $1-p$.

\begin{equation}
{(1-w^{n})}^k < 1-p
\Rightarrow
k > \frac{\log(1-p)}{\log(1-w^n)}
\end{equation}

\subsubsection{Cons and Pros}
TODO Here.

\subsection{Hough Transform}
其实就是把一条直线从实际空间的表示转换到参数空间的表示.但是如果存在垂直的直线,可能需要考虑使用极坐标来作为参数空间.

\begin{figure}[htbp]
    \centering
    \includegraphics[width=0.8\textwidth]{figures/hough1.png}
    \caption{Hough Transform w/o Noise}
\end{figure}

\begin{figure}[htbp]
    \centering
    \includegraphics[width=0.8\textwidth]{figures/hough2.png}
    \caption{Hough Transform w/ Noise and Outliers}
\end{figure}

题外话: 传统计算机视觉基于 module-based methods, 应用了多种技巧提升鲁棒性, 但现实生活中有很多复杂的特殊情况, 
因而需要大量的 rules or post-processing. 而深度学习则是 end-to-end 的方法, 但实际应用中是二者的取舍与均衡.
	\section{Keypoint Detection}

关键点检测常常应用于特征匹配, 例如在相机校准时, 需要检测棋盘格的角点. 
\subsection{What points are keypoints?}
\begin{itemize}
    \item Repeatablility: 在不同的图像中能够独立找到相同的点.
    \item Saliency: interesting points.
    \item Accuracy: 精确的位置.
    \item Quantity: 有足够多的点.
\end{itemize}

因而我们要求 detector 在光照, 视角, 尺度变化下具有不变性(invariance).

在这里我们主要讨论 Corner Detection.

\subsection{The Basic Idea of Harris Corner}

\begin{figure}[htbp]
    \centering
    \includegraphics[width=0.8\textwidth]{figures/window_moving.png}
    \caption{移动窗口-Sliding Window}    
\end{figure}

Move a window and explore intensity changes within the window.

Corner: significant change in all directions.

\subsection{Harris Corner}

一个 window,给定它的移动方向 $(u,v)$, 计算沿着这个方向的变化. 定义如下的 notation:
\begin{itemize}
    \item \textbf{Rectangle Window Function:} $w(x,y)$ (1 in the window, 0 outside)
    \[
        w(x,y)=\begin{cases}
            1 & \text{if } -b < x,y < b\\
            0 & \text{otherwise}
        \end{cases} \quad \text{where } b \text{ is the half width of the window}
    \]
    Then the rectangle window function when the center is at $(x_0,y_0))$ is $w'_{x_0,y_0}(x,y) = w(x-x_0,y-y_0)$.
    \item \textbf{Intensity Function:} $I(x,y) \implies$ Moving $(u,v)$, the square intensity difference within the window is 
    \[
        E_{x_0,y_0}(u,v)=\sum_{(x,y) \in N}[I(x+u,y+v)-I(x,y)]^2, \quad \text{where } N \text{ is the neighborhood of } (x_0,y_0)
    \] 
\end{itemize}
\begin{align*}
    E_{x_0,y_0}(u,v) &= \sum_{(x,y) \in N}w(x-x_0,y-y_0)[I(x+u,y+v)-I(x,y)]^2\\
    &= \sum_{x,y}w'_0(x,y)[I(x+u,y+v)-I(x,y)]^2 := \sum_{x,y}w'_0(x,y)D_{u,v}(x,y) \\
    &= \sum_{x,y} w(x-x_0,y-y_0) D_{u,v}(x,y) = \sum_{x,y} w(x_0-x,y_0-y) D_{u,v}(x,y) \\
    &= w \ast D_{u,v}
\end{align*}
Apply first-order Taylor expansion, we have 
\begin{equation}
    \renewcommand{\arraystretch}{0.85}
    \begin{aligned}
    D_{u,v}(x,y) &= [I(x+u,y+v)-I(x,y)]^2 \approx (I(x,y)+uI_x+vI_y-I(x,y))^2\\
    &= (uI_x+vI_y)^2 = \begin{bmatrix} u & v \end{bmatrix} \begin{bmatrix} I_x^2 & I_xI_y \\ I_xI_y & I_y^2 \end{bmatrix} \begin{bmatrix} u \\ v \end{bmatrix} \\
    \end{aligned}
\end{equation}
Then we have
\begin{equation}
    \renewcommand{\arraystretch}{0.85}
    \begin{aligned}
    E_{x_0, y_0}(u,v) &= w \ast \begin{bmatrix} u & v \end{bmatrix} \begin{bmatrix} I_x^2 & I_xI_y \\ I_xI_y & I_y^2 \end{bmatrix} \begin{bmatrix} u \\ v \end{bmatrix} \\
    &= \begin{bmatrix} u & v \end{bmatrix} \begin{bmatrix} w \ast I_x^2 & w \ast I_xI_y \\ w \ast I_xI_y & w \ast I_y^2 \end{bmatrix} \begin{bmatrix} u \\ v \end{bmatrix} \\
    (R \text{ is orthogonal })&= \begin{bmatrix} u & v \end{bmatrix} R^{-1} \begin{bmatrix} \lambda_1 & 0\\ 0 & \lambda_2 \end{bmatrix} R \begin{bmatrix} u \\ v \end{bmatrix} \\
    &= \lambda_1 u_R^2 + \lambda_2 v_R^2
    \end{aligned}
\end{equation}

那么 energy landscope 为一个椭圆, 两个特征值 $\lambda_1, \lambda_2$ 为椭圆的长短轴, 分别为能量变化最大和最小的方向.
\begin{figure}[htbp]
    \centering
    \begin{minipage}[t]{0.4\textwidth}
        \centering
        \includegraphics[width=0.8\textwidth]{figures/energy_landscope.png}
        \caption{Energy Landscape}
    \end{minipage}
    \begin{minipage}[t]{0.5\textwidth}
        \centering
        \includegraphics[width=0.9\textwidth]{figures/corner_map.png}
        \caption{特征值大小和点种类的关系}
    \end{minipage}
\end{figure}

根据这两个特征值的大小可以判断这个点是不是角点. 这个点是角点一般需要满足:

\begin{itemize}
    \item $\lambda_1, \lambda_2>b$
    \item $\frac{1}{k}<\frac{\lambda_1}{\lambda_2}<k$
\end{itemize}

一个快速的判断公式:

\begin{equation}
\begin{aligned}
\theta&=\frac 12(\lambda_1\lambda_2-\alpha(\lambda_1+\lambda_2)^2)+\frac12(\lambda_1\lambda_2-2t)\\
&=\lambda_1\lambda_2-\alpha(\lambda_1+\lambda_2)^2-t\\
&=\det(R)-\alpha\text{Trace}(R)^2-t
\end{aligned}
\end{equation}

其中 $\alpha\in[0.04,0.06], t\in[0.01,0.03]$.

Harris Corner 对平移和图像旋转是 equivariant的,对尺度变换 (Scale) 不是 equivariant 的.

\textbf{Choices of Window Function:} 由于 Rectangle window function 不具有旋转不变性, 可以使用 Gaussian window function(Isotropic "soft" window) 代替, 即
\[
    g_\sigma(x,y)=\frac{1}{2\pi\sigma^2}\exp\left( -\frac{x^2+y^2}{2\sigma^2} \right) \implies M(x_0, y_0) = \begin{bmatrix}
        g_\sigma \ast I_x^2 & g_\sigma \ast I_xI_y \\ g_\sigma \ast I_xI_y & g_\sigma \ast I_y^2 
    \end{bmatrix}
\]
此时, (记 $g_\sigma \ast \cdot = g(\cdot)$)
\begin{align*}
    \theta(x_0, y_0) &= \det(M(x_0, y_0)) - \alpha \text{Trace}(M(x_0, y_0))^2 - t \\
    &= \left(g(I_x^2) g(I_y^2) - g(I_xI_y)^2\right) - \alpha \left(g(I_x^2) + g(I_y^2)\right)^2 - t.
\end{align*}

\textbf{In Summary:} The whole process of Harris Corner Detection is as follows:
\begin{figure}[htbp]
    \centering
    \includegraphics[width=0.8\textwidth]{figures/Harris_process.png}
    \caption{Harris Corner Detection}
\end{figure}
\subsection{equivariant V.S. invariant}

等变 (equivariant): $F(TX)=T(F(x))$,对于translation和rotation是等变的.

不变 (invariant): $F(T(X))=F(X)$,也就是对于不同位置导出的角点还是那样,所以其实我们想要的是等变,也就是对于不同位置导出的角点做了同样的变化.

\subsection{How to prove Harris detector is equivariant?}

只要说明角点检测函数也是equivariant即可.

角点检测函数包括了求导和卷积两个操作,显然求导是equivariant的,因为导数会随着trans和rot做相同的变化.

很有趣的是卷积也是equivariant的:当你的filter function是各向同性的,那么这个卷积就是equivariant的;但是如果是一个椭圆形的window,那这个卷积就不是equivariant的了.

\begin{figure}[htbp]
    \centering
    \includegraphics[width=0.6\textwidth]{figures/light_invariant.png}
    \caption{Illumination invariant}
\end{figure}

这个高光说明不是环境 Illumination Invariant的

\subsection{How to do NMS with corner-response-function?}

一个简单的想法:

先给出一个阈值,把所有response排序,成为一个list,从上到下按顺序把这个pixel周围的大于阈值的踢出list.
这个跟之前的NMS区别在于之前需要一条边,现在只需要一个点,那么现在比之前踢出的像素点更多.

\subsection{Scale Invariant Detectors}

Harris-Laplacian, SIFT (Lowe)
	\include{04.CNN}
	\include{05.CNN_training}
	\include{06.CNN_improvement}
	\include{07.classification}
	\include{08.CNN_for_classification}
	\include{09.segmentation}
	\include{10.3Dvision}
	\include{11.camera_carlibration}
	\include{12.single_view_geometry}
	\include{13.epipolar_geometry}
	\include{14.3D_data}
	\include{15.3D_deep_learning}
	\include{16.Sequential_Modeling}
	\include{17.video_analysis}
	\include{18.Transformer}
	\include{19.object_detection_and_instance_segmentation}
	\include{20.generative_model}
	\include{21.pose_and_motion}
	\include{22.Instance_Level_6D_Object_Pose_Estimation}
	\include{23.motion}
	\include{24.Embodied_AI}
	\include{25.Summary_of_Computer_Vision}

	\clearpage
	% appendix: appendix-QRDecomposition, condition-number, transformation-in-space
	\appendix
	\section{Single Value Decomposition}
在线性代数或高等代数课程中我们都接触过奇异值分解, 这一部分将重点关注
SVD的几何意义, 主要参考了 \cite{SVD_ref_1}, \cite{SVD_ref_2} 以及 林晓疏学长信息科学中的数学的课程notes.

\subsection{Introduction}
对于实对称方阵 $\mathbf{M}_{n\times n}$, 我们有熟知的特征值分解:
$\mathbf{M} = \mathbf{Q} \mathbf{\Lambda} \mathbf{Q}^{-1}$ 其中 $\mathbf{Q}$ 是正交矩阵, $\mathbf{\Lambda}$ 是对角矩阵. 
而对于更一般的矩阵 $\mathbf{A}_{m\times n}$, 我们的目标是寻找矩阵 $\mathbf{U}_{m\times r} \Sigma_{r\times r} \mathbf{V}_{r\times n}$ 使得
\[
\mathbf{A} = \mathbf{U} \Sigma \mathbf{V}^T
\]
其中 $\mathbf{U}$ 和 $\mathbf{V}$ 是正交矩阵, $\Sigma$ 是对角矩阵且元素均为正实数. 

\marginpar{\kaishu $\mathbf{U,\Sigma,V}$ 的维度有时会有所不同, 另一种常见的形式是 $\mathbf{U}_{m\times m}$, $\Sigma_{m\times n}$, $\mathbf{V}_{n\times n}$.}

TODO Here.
	\include{condition-number}
	\include{transformation-in-space}
	\include{DOF_and_rank}	
	\include{appendix-QRDecomposition}

    \bibliographystyle{plain}
    \bibliography{CV_notes}
\end{document}